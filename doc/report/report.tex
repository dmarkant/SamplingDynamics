\documentclass[english,doc]{apa}
\usepackage[T1]{fontenc}
\usepackage[latin9]{inputenc}
\usepackage{color}
\usepackage{amsmath}
\usepackage{amssymb}
\usepackage{graphicx}
\usepackage{babel}
\usepackage{setspace}
\doublespacing

\makeatletter
%%%%%%%%%%%%%%%%%%%%%%%%%%%%%% Textclass specific LaTeX commands.
\helvetica
\author{.} % hack around some bugs in apa.cls
\affiliation{.} % hack around some bugs in apa.cls

%%%%%%%%%%%%%%%%%%%%%%%%%%%%%% User specified LaTeX commands.
%% uncomment the following line to get this information
% below the tile
\note{Draft of \today}

%% Some information on the journal. This will be
%% printed on the headline of the first page
%% Just uncomment wir a '%' if you don't want this
% Journal name
%\journal{The Test Journal}
% volume, number, pages
%\volume{X}
% copyright notice
%\ccoppy{\textcopyright\ The Author}
% Serial number or other reference
%\copnum{ISSN XXX-XXXX-XXXX}

%% The usual ...
%\acknowledgements{ \ldots}

\makeatother

\setlength\parindent{0pt}

\begin{document}

\title{ Sequential Sampling Models for DFE}


\rightheader{Report Modeling DFE}


\author{.} 
\affiliation{.}
\maketitle
{}
%\clearpage



\section{Basic model}
\noindent The basic model is a sequential sampling model with two absorbing boundaries as shown in Figure \ref{f:1Wiener}. Each trajectory (the colored curves) represents the process in time for choosing  either option A or option B. It is continuous in time and does NOT represent the number of draws. The choice options are defined below. 

%\begin{figure}[ht]
%\begin{center}
%\includegraphics[scale=0.4]{Model_1Wiener_DFE} \caption[ja]{Three trajectories for the basic underlying process. In this example the drift rate $\mu$ is positive, i.e., option $A$ is chosen over option $B$ more often}\label{f:1Wiener}
%\end{center}
%\end{figure}

\noindent For all model versions we assume at least two processes, one for choosing option A and one for choosing option B. The structure of the gambles is mapped onto the drift rate  parameters $\mu_A$ and $\mu_B$. 

\section{Structure of the gambles}
\noindent I refer to Doug's output file.

\noindent Each of the gambles have a different expected value and a different variance. 
\begin{itemize}
\item For the High (H) condition the expected value ranges from about 5.8 to 6.2 in the gain domain and from $-4.8$ to $-5.2$ in the loss domain.   

\item For the Low (L) condition the expected value ranges from about 4.8 to 5.2 in the gain domain and from $-5.8$ to $-6.2$ in the loss domain. 

\item The low variances range from about 10 to 30 and the high variance between 60 and 90.


\item The difference in expected values between H and L ranges from .71 to 1.3.

\item The values actually drawn vary between $-24$ and 24.
\end{itemize}

\noindent Except for the variances and the actual values the ranges are relatively small. 

\noindent The question is what to map onto the drift rates. We want to weight the input (the parameters in the model) somehow.  There are too many different expected values (altogether 166) to even consider this as an option. To consider the difference in expected values is not reasonable since both gambles are not directly compared (at least not for the low switchers but we will use that to construct the model for decisions from descriptions). And they would be too many as well (83).  To use the actual values drawn could be one way to go. The  ids for the set of outcomes are between 0 and 45 for the L condition and between 2 and 45 between the H condition. That is we have max 46 different values (hopefully less). 
 
\section{Specific models}
\noindent Using the actual values drawn gives one possible input for the L condition to model:
 
\begin{equation} 
\mu_{L_{id_n}} = (w_{i} L_{x1n} + w_{j} L_{x2n} + w_{k} L_{x3n}) / \sqrt{Var(L_n)}  \label{EQ:mu1}
\end{equation}

\noindent The same holds for the H condition. Implicitly we assume a zero input for H when L is considered and zero input for L when H is considered. $w_i$, $w_j$, and $w_k$ are the parameters to be estimated from the data. With max 46 different values this results in 46 different $w$. 

An alternative input is 

 
\begin{equation} 
\mu_{L_{id_n}} = (w_{i} L_{x1n} + w_{j} L_{x2n} + w_{k} L_{x3n})/ para(Var(L))  \label{EQ:mu2}
\end{equation}

\noindent The difference between both inputs  is that in Eq. \ref{EQ:mu1}  the standard deviation from the actual gambles are taken whereas in  \ref{EQ:mu2} low and high variances build two clusters and each is presented by an additional parameter. There are certainly more ways to set up the drift rate, e.g., by combining the values in the gambles in a reasonable way or by finding a different model input structure. We should discuss it.  


 
Whatever the input to the drift is there are also various model architectures we have  discussed briefly (there are certainly more, but let's start with these). For simplicity the processes are shown in the following without the trajectories.  


\subsection{Model 1}
\noindent We assume one process for both options H and L with two subprocesses. In one subprocess H is considered and in the other subprocess L is considered. 

%\begin{figure}[ht]
%\begin{center}
%\includegraphics[scale=0.4]{Schema_serial.pdf}
%\end{center}
%\end{figure}

\noindent $t_1$ indicates when the DM switched from H, say, to L.  In the model $t_1$ is a RV (e.g., Uniform with small variance, i.e., almost fixed or with high variance which approaches a geometric distribution) and $t_1$ may be even related to the average number of draws. Low switchers switching happens once; f high switchers often. This is mapped one to one to the process: for low switchers we assume one switch and for frequent switchers the processes go back and forth. The sequence in which the processes are considered (first H and the L or vice versa) matters for the prediction (not for the choice probabilities but for the mean RT). We could also assume that the DM start randomly with H or L. Note that the starting process of the second process is not exactly as suggested in the figure. The trajectories are random (see \ref{f:1Wiener} and so are the next starting positions. 

\subsection{Model 2}
\noindent The second model assumes two separate processes, one for H and one for L. However, they are not directly interlinked. Note that the choice options are now accept/reject H and accept/reject H.  Also note, that the boundaries may be different in size. 

%\begin{figure}[ht]
%\begin{center}
%\includegraphics[scale=0.4]{Schema_mixture.pdf}
%\end{center}
%\end{figure}
Both processes are combined by assuming a mixture of processes. With probability $p$  H is processes and with probability $(1-p)$ L is processed. $p$ is a model parameter but may be directly linked to the number of draws taken from each option. I don't know yet how to model the switching behavior for this model. 
\section{Remarks}
\begin{itemize}
\item Most importantly to get it started is to decide what features of the gambles should be  map onto the model parameters. 
\item Risk attitude can be mapped onto a approach/avoidance parameter like in DFT (cf. forgetting). That is, the drift rate in Eqs. \ref{EQ:mu1} and \ref{EQ:mu1} are extended by $-\gamma x$, where $\gamma$ is a model parameter and $x$ is a state in the state space (roughly the evidence to accumulate). 
\end{itemize}



\section{Exploratory analysis}

Additional analyses were conducted to answer the following questions:

\begin{itemize}
\item Are overall measures like sample size and number of switches related to subject- and problem-level factors?
\item Can trial-by-trial decisions (whether to continue sampling the same option, to switch to the other option, or to stop sampling altogether) be predicted by changes in the evidence accumulated?
\end{itemize}


\subsection{Subject- and problem-level analysis}

The following predictors were considered:

\begin{itemize}
\item Group (young/old)
\item Session (1-21)
\item Domain (gain/loss)
\item Pairtype (HH-LL or HL-LH): all option pairs included one high-variance and one low-variance option. For HH-LL pairs, the H option had high variance (this was predicted to be a difficult problem); in HL-LH pairs, the H option had low variance (predicted to be an easy problem).
\item EV-diff: difference in expected value between H and L option
\item Total variance (combined between two options; although all pairs had a high and low variance option, there was still some variability in the total variance across different pairs)
\end{itemize}


\subsubsection{Number of switches}

Ran mixed effects linear regression on the number of switches, with the variables above as fixed effects, and a random effect term for sample size by participant. 
This random effect leads to a different coefficient for each participant describing the effect of samples on the number of switches. 
For frequent switchers this value is high (since number of switches is highly correlated with number of samples), while for people who switch the same number of times regardless of sample size it should be at zero.

There were no significant effects of any of the variables above on the number of switches.
This result was consistent with earlier analyses that indicated 1) a bimodal distribution of switching rates, and 2) stable switching rates within subjects.
For following analyses, people were divided into two groups based on their switching rate (FREQ and RARE switchers) using a median split on the average switching rate.


\subsubsection{Sample size}

The next question was how the total sample size was related to the same set of predictors (now adding the switching group variable based on median split).
There was a significant effect of the switch group (FREQ/RARE; $p<.001$), with frequent switchers taking fewer samples than rare switchers. 
There was also an effect of session ($p<.001$), with the number of samples decreasing over time.
Additional modeling indicated that this decrease in sample size was specific to the RARE group. 
However, there were no other effects of group or problem on sample size.

In sum, the results suggest that after accounting for variability between participants, the features of the problem did not have a significant impact on switching rate or overall sample size.


\subsection{Trial-by-trial analysis}

Although the preceding analysis didn't reveal any significant relationships between features of the problem and sample size or overall switching rate, it may be that trial-by-trial changes in a participant's experience were related to when they made decisions to switch or stop.

The following trial-by-trial variables were considered:

\begin{itemize}
\item Sample mean
\item Deviation of outcome from sample mean
\item Absolute deviation from sample mean
\item Sample variance
\item Difference in sample means (current option - other option; note that when one option has not yet been sampled, its mean is assumed to be the average of all option EVs in the same domain)
\end{itemize}


\subsubsection{Stay vs. leave}

I first compared decisions to STAY (continue sampling the same option) or LEAVE (\emph{either} switch to a different option or stop sampling).
Since frequent switchers chose to LEAVE on nearly every trial, here the focus was on rare switchers only.
The following table shows the result of a logistic regression model (STAY=0, LEAVE=1) with the variables above as predictors.
\footnote{Note that separate analyses using rank correlations led to the same overall pattern of results as shown in the tables.}


\begin{center}
\begin{table}[htdp]
\caption{Stay vs. Leave (RARE only)}
\begin{tabular}{l|c|c|}
& $\beta$ & $p$ \\
\hline
Sample mean & -.007 & .002 \\
Deviation of outcome from sample mean & .005 & .006 \\
Absolute deviation from sample mean & .05 & <.001 \\
Sample variance & -.01 & <.001 \\
Difference between sample means & -.002 & .35
\end{tabular}
\label{default}
\end{table}%
\end{center}

LEAVE decisions were more likely on trials with lower sample variance (but this is not surprising since in the RARE group these decisions occur at the end of longer streaks of sampling a single option).
They were also more common when the sample mean was low, and when the most recent outcome deviated strongly from the sample mean.
As compared to STAY decisions, the likelihood of LEAVE decisions did not vary based on the difference in sample means for RARE switchers.

\subsubsection{Switch vs. stop}

The second comparison focused on the subset of trials in which LEAVE decisions were made, comparing SWITCH and STOP decisions.
Analyses were performed separately on RARE and FREQ groups of participants.

\begin{center}
\begin{table}[htdp]
\caption{Switch vs. Stop (RARE switchers)}
\begin{tabular}{l|c|c|}
& $\beta$ & $p$ \\
\hline
Sample mean & .008 & .06 \\
Deviation from sample mean & .007 & .07 \\
Absolute deviation from sample mean & -.007 & .23 \\
Sample variance & 0. & .76 \\
Difference between sample means & 0. & .95
\end{tabular}
\label{default}
\end{table}%
\end{center}

\begin{center}
\begin{table}[htdp]
\caption{Switch vs. Stop (FREQ switchers)}
\begin{tabular}{l|c|c|}
& tau & p \\
\hline
Sample mean & .002 & .56 \\
Deviation from sample mean & .02 & <.001 \\
Absolute deviation from sample mean & .05 & <.001 \\
Sample variance & -.008 & <.001 \\
Difference between sample means & .02 & <.001
\end{tabular}
\label{default}
\end{table}%
\end{center}


For RARE switchers, decisions to STOP sampling were more likely when the sample mean of the current option was high and when the most recent outcome was high relative to the sample mean (although both effects are marginal).
As in the comparison of STAY/LEAVE decisions, the difference in sample means was not associated with an increased likelihood of a STOP decision.

For FREQ switchers, STOP decisions were not related to the current sample mean, but were more likely following outcomes that deviated from the sample mean.
STOP decisions were also more likely when the current option was higher than the other option.



\end{document}